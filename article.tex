\documentclass[11pt,a4paper]{article}


\usepackage{xpatch} % for overwriting commands
\usepackage{sectsty}
\usepackage{graphicx} % for inserting images

% for multiple images in one figure
\usepackage{subcaption}

% to wrap text arround a figure
\usepackage{wrapfig}

% implement example in doc https://www.overleaf.com/learn/latex/Positioning_images_and_tables#Positioning_images


\usepackage[ngerman]{babel} % sets document language to german
\usepackage[citestyle=authoryear]{biblatex} % imports biblatex package

\usepackage[]{fancyhdr} % custom headers and footers

\usepackage{hyperref} % for clickable ToC
\hypersetup{
    colorlinks,
    citecolor=black,
    filecolor=black,
    linkcolor=black,
    urlcolor=black
} % makes links in ToC black

\addbibresource{main.bib}

% margins
\topmargin=-0.45in
\evensidemargin=0in
\oddsidemargin=0in
\textwidth=6.5in
\textheight=9.0in
\headsep=0.25in

% codeblocks
\usepackage{listings}
\usepackage{xcolor}
\definecolor{codegreen}{rgb}{0,0.6,0}
\definecolor{codegray}{rgb}{0.5,0.5,0.5}
\definecolor{codepurple}{rgb}{0.58,0,0.82}
\definecolor{backcolour}{rgb}{0.95,0.95,0.92}
\lstdefinestyle{codeStyle}{
    backgroundcolor=\color{backcolour},
    commentstyle=\color{codegreen},
    keywordstyle=\color{magenta},
    numberstyle=\tiny\color{codegray},
    stringstyle=\color{codepurple},
    basicstyle=\ttfamily\footnotesize,
    breakatwhitespace=false,
    breaklines=true,
    captionpos=b,
    keepspaces=true,
    numbers=left,
    numbersep=5pt,
    showspaces=false,
    showstringspaces=false,
    showtabs=false,
    tabsize=2
}
\lstset{style=codeStyle}

% Variables
\usepackage{amsmath}

\def\exerciseDate{Übungsdatum}
\def\turnInDate{Abgabedatum}
\def\authors{Autor}
\def\titleVar{Titel}
\def\subject{Unterricht x.Jhg}

% Sets Headers and Footers
\pagestyle{fancy}
\fancyhead{} % Clear default header and footer
\fancyfoot{}
\fancyhead[L]{\subject} % Left header
\fancyhead[C]{} % Center header
\fancyhead[R]{\turnInDate} % Right header
\fancyfoot[L]{\authors}
\fancyfoot[R]{Seite \thepage}

\begin{document}

% title
    \title{\titleVar}
    \author{\authors}
    \date{Übungsdatum: \exerciseDate}

% titlepage
    \begin{center}
        \includegraphics[width=7cm]{LogoITHTL_white.png}
        \textbf{{\let\newpage\relax\maketitle}}
        \subject
        \vspace{5px}
        \\Abgabedatum: \turnInDate
        \vspace{15px}
        \\Bewertung:
    \end{center}
    \thispagestyle{empty}
    \pagebreak

% optional TOC
    \tableofcontents
    \listoffigures
    \pagebreak

% content
    \section{section}
    content
    \subsection{subsection}
    content
    \subsubsection{subsubsection}
    content

    \paragraph{codeblocks}

    \begin{lstlisting}[language=Python, caption=Python example, label=example]
    import numpy as np

    def incmatrix(genl1,genl2):
        m = len(genl1)
        n = len(genl2)
        M = None #to become the incidence matrix
        VT = np.zeros((n*m,1), int)  #dummy variable

        #compute the bitwise xor matrix
        M1 = bitxormatrix(genl1)
        M2 = np.triu(bitxormatrix(genl2),1)

        for i in range(m-1):
            for j in range(i+1, m):
                [r,c] = np.where(M2 == M1[i,j])
                for k in range(len(r)):
                    VT[(i)*n + r[k]] = 1;
                    VT[(i)*n + c[k]] = 1;
                    VT[(j)*n + r[k]] = 1;
                    VT[(j)*n + c[k]] = 1;

                    if M is None:
                        M = np.copy(VT)
                    else:
                        M = np.concatenate((M, VT), 1)

                    VT = np.zeros((n*m,1), int)

        return M
    \end{lstlisting}


    \paragraph{paragraph}
    this is a reference to a figure (Abb. \ref{fig:demo-figure})

    \begin{figure}[h]
        \centering
        \includegraphics[width=1cm]{LogoITHTL_white.png}
        \caption{this is a figure}
        \label{fig:demo-figure}
    \end{figure}

\end{document}
